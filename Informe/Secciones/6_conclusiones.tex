\paragraph{}
Se observa en el presente trabajo la utilidad de los filtros adaptativos como sistemas de identificación. Mediante un algoritmo iterativo, se puede obtener un filtro cuyos coeficientes se asemejan a los del modelo del sistema físico real. Se observa como, con el paso del tiempo, el filtro converge a los valores correctos, mejorando el error en cada iteración.
\paragraph{}
Sin embargo, también se observa como el ruido de medición puede afectar severamente la convergencia del algoritmo, y por lo tanto, los resultados finales. Si el mismo es demasiado alto, los coeficientes no convergerán a los valores correctos.
\paragraph{}
Por otro lado, se observa la capacidad del algoritmo de adaptarse a la situación presente. Como por ejemplo, cuando se corta el resorte, los coeficientes modifican su tendencia, dirigiéndose a los nuevos valores correctos. Se observó también, que la velocidad de convergencia del algoritmo puede ser afectada severamente cuando el resorte se corta.

