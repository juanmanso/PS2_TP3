%Preambulo para articulo científico de LaTeX

\usepackage[a4paper,left=3cm,right=3cm,bottom=3.5cm,top=3.5cm]{geometry} 	% Configuro la geometría del papel
%\usepackage{microtype}								% Mejora el "spacing" de las palabras
\usepackage[spanish]{babel} 							% Compatibilizo los signos del español
	\addto\captionsspanish{\renewcommand{\tablename}{Tabla}}		%% Redefino nombres preestablecidos por Babel
	\addto\captionsspanish{\renewcommand{\listtablename}{Índice de tablas}}	%% y así en vez de Cuadro dirá Tabla.
\usepackage{amsmath, amsfonts, amssymb}						% Entornos matemáticos, fuentes y símbolos
\usepackage{graphicx}								% Necesario para insertar figuras
\usepackage{fancyhdr}								% Para manipular headers y footers
\usepackage[usenames,dvipsnames]{color}						% \color{color deseado} {lo que querés que tenga color}
\usepackage{subcaption}								% Permite captions del tipo 1a, 1b
\usepackage{multirow}								% Para tablas
\usepackage{float}

% Para video
\ifVideo
	\usepackage{media9}
	\addmediapath{./../reportes/}
\fi

%\usepackage{times}
%\usepackage{mathtools}
%\usepackage{upgreek} % letras griegas sin cursiva
%\usepackage{cancel}
\usepackage{rotating}
\usepackage{tikz}
\usepackage{pgfplots}
%	\pgfplotsset{compat=1.12}
	\usetikzlibrary{plotmarks}% matlab2tikz
\usepackage{grffile}% matlab2tikz 
	\usetikzlibrary{calc,patterns,decorations.pathmorphing,decorations.markings}

\ifListings
	\usepackage{listings}

	\providecommand{\lstinputpath}[1]{\lstset{inputpath=#1}}

%	\input{.lst_default.tex}
	\input{.lst_matlab.tex}
%	\input{.lst_c.tex}
%	\input{.lst_c++.tex}
	
% 	\input{.lst_pseudocode.tex}


\fi

\ifSiunitx
\usepackage{siunitx}											% Unidades: \SI {cantidad} {\unidad} (necesita texlive-science)
	\sisetup{load-configurations = abbreviations}							% Habilita poner \cm en vez de \centi\metre
	\sisetup{output-decimal-marker = {,}}									% Cambia los puntos decimales por comas
	\sisetup{per-mode = fraction}											% Pone las unidades como fracción
	\sisetup{quotient-mode = fraction}										
\fi


\ifTodonotes
\usepackage{xargs}
\usepackage[colorinlistoftodos,prependcaption,textsize=tiny]{todonotes}


	\newcommandx{\Juan}[2][1=]{\todo[linecolor=blue,backgroundcolor=blue!25,bordercolor=blue,#1]{#2}}
	\newcommandx{\Mati}[2][1=]{\todo[linecolor=green,backgroundcolor=green!25,bordercolor=green,#1]{#2}} % OliveGreen
	\newcommandx{\Emi}[2][1=]{\todo[linecolor=Plum,backgroundcolor=Plum!25,bordercolor=Plum,#1]{#2}}
	\newcommandx{\unsure}[2][1=]{\todo[linecolor=red,backgroundcolor=red!25,bordercolor=red,#1]{#2}}
	\newcommandx{\thiswillnotshow}[2][1=]{\todo[disable,#1]{#2}}
\fi


\usepackage{booktabs}														% Permite hacer tablas sin separadores en el medio
\usepackage{placeins}														
		\let\Oldsection\section												%% Permite que los flotantes (como figuras) no aparescan
	\renewcommand{\section}{\FloatBarrier\Oldsection}						%% antes o después de su sección correspondiente.
		\let\Oldsubsection\subsection
	\renewcommand{\subsection}{\FloatBarrier\Oldsubsection}		
		\let\Oldsubsubsection\subsubsection
	\renewcommand{\subsubsection}{\FloatBarrier\Oldsubsubsection}
\usepackage{hyperref}														% Debe ser agregado al final del preambulo

\hypersetup
{    bookmarks=true,         % show bookmarks bar?
     unicode=false,          % non-Latin characters in Acrobat’s bookmarks
     pdftoolbar=true,        % show Acrobat’s toolbar?
     pdfmenubar=true,        % show Acrobat’s menu?
     pdffitwindow=false,     % window fit to page when opened
     pdftitle={\myTitle},    		 % title
     pdfauthor={\myAuthorSurname},   % author
	 pdfcreator={\myAuthorSurname},	 % creator = author
     pdfsubject={\mySubject},		 % subject of the document
     pdfkeywords={\myKeywords},
     colorlinks=true,        % false: boxed links; true: colored links
     linkcolor=black,        % color of internal links (change box color with linkbordercolor)
     citecolor=black,        % color of links to bibliography
     filecolor=magenta,      % color of file links
     urlcolor=cyan           % color of external links
}

%Configuro la pagina con los encabezaos y pies de paginas
\pagestyle{fancy}										% Para agregar encabezados y pie de paginas	
\lhead{\mySubject}										% Encabezado izquierdo
\rhead{\includegraphics[scale=0.15]{\myHeaderLogo}} 	% Encabezado derecho (logo de la FIUBA)	
\ifPutgroup
\chead{\texttt{Grupo Nº\myGroupNumber} }%\\ \textit{\footnotesize{\myTimePeriod}}}
\fi				
