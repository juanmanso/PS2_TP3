%% Este archivo contiene las funciones auxiliares para escribir en LaTeX
%% Dichas funciones resuelven la sintaxis de generar figuras, por ejemplo,
%% dejando el código más compacto y facilitando la corrección del mismo.



% Comando para graficar eps. 1er arg, escala. 2do, ruta. 3ro, caption. 4to, label.
\providecommand{\HgraficarEPS}[4]{
			\begin{figure}[h!]
				\centering
					\scalebox{#1}{\input{#2}}
					\caption{#3}
					\label{#4}
			\end{figure}

}

\providecommand{\HgraficarPNG}[4]{
			\begin{figure}[h!]
				\centering
					\includegraphics[scale=#1]{#2}
					\caption{#3}
					\label{#4}
			\end{figure}

}


% Comando para graficar eps en el lugar previsto.
\providecommand{\graficarEPS}[4]{
			\begin{figure}[h]
				\centering
					\scalebox{#1}{\input{#2}}
					\caption{#3}
					\label{#4}
			\end{figure}

}

\providecommand{\graficarPNG}[4]{
			\begin{figure}[h]
				\centering
					\includegraphics[scale=#1]{#2}
					\caption{#3}
					\label{#4}
			\end{figure}

}

\providecommand{\graficarPDF}[3]{
			\begin{figure}[h!]
				\centering
		\includegraphics[width=1.0\textwidth,keepaspectratio]{#1}
					\caption{#2}
					\label{#3}
			\end{figure}

}


\providecommand{\graficarPDFwide}[3]{
			\begin{figure}[h!]
				\centering
		\includegraphics[scale=0.5,trim={6,5cm 0 0 0}]{#1}
					\caption{#2}
					\label{#3}
			\end{figure}

}

\providecommand{\graficarPDFa}[4]{
			\begin{figure}[h!]
				\centering
				\includegraphics[scale=0.5,trim={#1}]{#2}
					\caption{#3}
					\label{#4}
			\end{figure}

}



\providecommand{\underuparrow}[2]{\underset{\underset{#2} \uparrow} #1 }

\providecommand{\cltext}[2]{\color{#1}{\huge{#2}}}

\providecommand{\cstext}[2]{\color{#1}{\large{#2}}}

\providecommand{\vect}[1]{\boldsymbol{#1}}
\providecommand{\dvect}[1]{\dot{\boldsymbol{#1}}}
\providecommand{\dd}{\mathrm{d}}
